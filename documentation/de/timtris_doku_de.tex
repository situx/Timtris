\documentclass[10pt,a4paper,titlepage]{article}
\usepackage[utf8]{inputenc}
\usepackage{amsmath}
\usepackage{amsfonts}
\usepackage{amssymb}
\usepackage{graphicx}
\usepackage{lmodern}
\usepackage{xcolor}
\usepackage{qtree}
\usepackage{listings}
\usepackage{color}
\usepackage[xetex, hypertexnames=false, breaklinks=true, pdfborder={0 0 0},
pdfauthor={Timo Homburg},
pdftitle={Timtris Projektbeschreibung},
pdfsubject={Timtris Projekt},
pdfkeywords={Java,Game,Tetris},
pdfproducer={Xetex with hyperref},
pdfcreator={Pdflatex}]{hyperref}
\definecolor{gray}{rgb}{0.4,0.4,0.4}
\definecolor{darkblue}{rgb}{0.0,0.0,0.6}
\definecolor{cyan}{rgb}{0.0,0.6,0.6}
\lstset{
basicstyle=\footnotesize,
frame=single,,
  columns=fullflexible,
  showstringspaces=false,
  commentstyle=\color{gray}\upshape,
  keepspaces
                 % Linienstaerke des Rahmens
}
\renewcommand{\contentsname}{Inhaltsverzeichnis}
\lstdefinelanguage{XML}
{
  morestring=[b]",
  morestring=[s]{>}{<},
  morecomment=[s]{<?}{?>},
  stringstyle=\color{black},
  identifierstyle=\color{darkblue},
  keywordstyle=\color{cyan},
  morekeywords={xmlns,version,type}% list your attributes here
}
\begin{document}
\tableofcontents
\ \\\\\\\
\textbf{\large Timtris Projektbeschreibung}\\
\section{Einleitung}
Das Spiel Timtris ist eine Java Implementierung der beiden Spiele Tetris und Dr. Mario welche auch über das Netzwerk spielbar ist. Es ist aus einem Universitätsprojekt entstanden.\\
Name der Klasse mit der Main-Methode: Board.class\\
\subsection{Spielziel und Spielfunktionen}
Das Spiel basiert auf dem bekannten Spiel Tetris.
7 verschiedene Steine sollen auf dem Spielfeld so angeordnet werden, dass sich volle
horizontale Reihen ergeben.
Diese werden nach Vervollständigung gelöscht und Punkte werden gutgeschrieben.
(1 Punkt pro gelöschtem GeometricObject, bei 2 oder mehr Reihen +4Punkte pro zusätzlich
gelöschter Reihe)
Verloren ist das Spiel sobald keine Figuren mehr auf das Spielfeld passen.
Im Singleplayermodus ist das Spiel theoretisch endlos, wird jedoch von Level zu Level
schneller und durch Effekte erschwert.
\subsection{Effekte im Spiel}
Die Effekte sind im einzelnen: 
\begin{itemize}
	\item Verschobene Startposition des herunterfallenden Objektes
	\item Rotation des Objektes beim Erstellen
	\item Das sofortige Herunterfallen des gerade erstellten
	Objektes an der Erstellungsposition
	\item Löschen eines zufälligen GeometricObjects in der
	Liste
	\item Hinzufügen einer neuen Zeile mit einer Lücke von unten
	\item Das Herunterfallen
	einer Figur mit 2 horizontalen GeometricObjects (siehe MarioMod)
\end{itemize}
Das Auftreten der Effekte ist von den Schwierigkeitseinstellungen unter Options abhängig.
Der Defaultwert bei Start des Spiels ist Medium.
\section{Netzwerkmodus}
Es existiert ein Netzwerkspielermodus mit den Möglichkeiten das Spiel über Netzwerk mit
einem Mitspieler ähnlich wie im Singleplayermodus zu spielen (theoretisch unendliches
Spiel bis ein Spieler keinen Stein mehr setzen kann).
Möglich ist außerdem ein Spiel auf Punkte in dem der Server vorher eine
Punktehöchstgrenze festsetzt.
Wird diese von einem der Spieler erreicht steht er als Gewinner fest.
Zudem kann das Netzwerkspiel auf Zeit gespielt werden.
Hierbei erfolgt die Ermittlung des Gewinners nach einer bestimmten Zeitspanne anhand der
erreichten Punkte.
\subsection{Limitierungen des Netzwerkmodus}
Der Netzwerkmodus ist nur zwischen 2 Mitspielern möglich.
Wird im Netzwerkmodus von einem Spieler eine Reihe gelöscht, so erscheint eine neue fast
vollständige Reihe auf dem Spielfeld des Gegners.
Theoretisch wäre das Spiel auf beliebig viele Clienten erweiterbar gewesen, jedoch erschien
das aufgrund der zu erwartenden Spielreihen die zu den Clienten hinzugefügt werden ohne
ein neues Konzept der Verteilung der Reihen auf die Mitspieler unvorteilhaft, da es zu viele
Reihen auf einmal pro Spieler bedeutet hätte.
\section{MarioMod}
Ein weiterer eingebauter Mod ist an das Spiel Dr. Mario angelehnt.
Hier gilt es die auf dem Spielfeld verteilten Kreise durch vertikale Anordnung von
mindestens 4 Spielsteinen gleicher Farbe (inklusive einem Kreis) zu löschen.
Erschwert wird dies dadurch dass, anders wie bei dem klassischen Dr. Mario Spiel evtl.
übrig bleibende Steine bei Löschung einer 4-er Reihe nicht herunterfallen und somit in der
Luft „schweben“ bleiben.
Werden die Steine so gesetzt das 2 Reihen gelöscht werden müssten, wird nur eine Reihe
gelöscht um das Spiel noch zusätzlich zu erschweren.
\subsection{Schwierigkeitsgrad}
Der Schwierigkeitsgrad erhöht sich in dieser Modifikation durch die Geschwindigkeit der
fallenden Steine, die sich nach einer bestimmten Anzahl generierter Blöcke erhöht.
(abhängig von der gewählten Schwierigkeit im Optionenmenü)
Sind alle Kreise auf dem Spielfeld gelöscht worden setzt sich die Geschwindigkeit wieder
zurück und ein neues Spielfeld mit mehr zu vernichtenden Kreisen wird generiert (das
nächste Level).
Verloren ist das Spiel auch hier wenn der nächste Stein nicht mehr auf das Spielfeld passt.
\section{Objektorientierte Architektur des Spiels}

Die objektorientierte Architektur des Spieles begründet sich zum einen in den schon im
Semester gefertigten GeometricObjects, die weiterverwendet und zu z.B. den Tetrisfiguren
zusammengesetzt werden.
Diese basieren auf der Klasse Vertex, die u.a. zur Bewegung der GeometricObjects
verwendet wird.
Auch die Unterklasse der GeometricObjects Circle wurde im Mod des Spieles
implementiert.
Weiterhin wurde die im Praktikum eingeführten Schnittstellen MoveableObject und
Paintable sowie ein KeyListener für die Steuerung implementiert.
Viele weitere Objekte wurden zudem aus den Standard java Libraries importiert (z.B:
Socket uva.).
\subsection{Klassenaufteilung}
Auch eine sinnvolle Aufteilung der Funktionalitäten des Programmes nach Klassen ist
erfolgt.
\begin{itemize}
	\item So wird die Klasse Board für das Spielfeld und seine Funktionen genutzt (z.b. das Zeichnen
	der Objekte, den Aufbau des Pausescreens u.a.).
	\item Die Deleteklasse kümmert sich um die Frage ob GeometricObjects gelöscht werden sollen
	und wenn es so ist übernimmt sie auch das Löschen der Objekte.
	\item Die Gameklasse startet und beendet die in Timtris implementierten Spiele und schreibt
	den GameLog mit den erreichten Punkteständen der jeweiligen Spiele in das Verzeichnis
	des Spieles.
	\item Die Klasse KeyMove beschäftigt sich mit den Links-Rechts-Bewegungen der Tetrisfiguren
	und eventuellen Crashes die durch die Bewegungen verursacht werden. Außerdem enthält
	sie die Funktion „down“ mit der die Figuren soweit wie möglich nach unten geschoben
	werden können.
	\item Die Klasse Makeo erzeugt die für das Tetrisspiel nötigen Figuren und je nach gewähltem
	Schwierigkeitsgrad etwaige Zusatzeffekte.
	\item In der MarioModklasse stehen die angepassten Funktionen für den MarioMod (z.B. zum
	generieren der Objekte, zum Löschen usw.)
	\item Die Klasse Menu beinhaltet einzig den Aufbau des Hauptmenüs (der MenuBar) und 2
	Frames (für about, Spielanleitung und Optionen) des Spieles und gibt die aufgebaute
	Menüleiste zum einbinden an die Klasse Board zurück.
	\item In der Networkklasse werden sowohl die Funktionen zum Starten des Clients als auch zum
	Starten des Servers und zum Erstellen der verschiedenen Frames und zum schließen der
	Netzwerkverbindungen gespeichert.
	\item Die Klasse Rotate enthält Funktionen um die Tetrisfiguren rotieren zu lassen, angepasst auf
	jede Tetrisfigur.
	\item Eine Unterklasse von JLabel stellt die StatusBar Klasse dar, die dazu dient die Statusanzeige
	bez. Level und Punkten sowie im Netzwerkmodus die Punkte des Gegners, die verbleibende
	Zeit und die Punktegrenze anzuzeigen.
\end{itemize}
Alle Funktionen der Klassen sind nicht statisch und können somit jederzeit ersetzt werden.
\section{Bekannte Fehler}
Leider war es mir nicht möglich folgenden Fehler zu korrigieren:
Er tritt im Netzwerkmodus auf wenn 2mal hintereinander der gleiche Spieler den Server
stellt und verhindert die Ausführung des Timers auf der Clientseite.
Wird der Server jedoch abwechselnd gestellt, so funktionieren beide Seiten.

\end{document}